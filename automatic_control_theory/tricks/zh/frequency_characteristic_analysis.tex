\documentclass{ctexart}
% 数学公式
\usepackage{amsmath, amssymb, mathtools}
%处理图片
\usepackage{graphicx}
\usepackage{float}
\usepackage{subfigure}
% 序号
\usepackage{enumerate}
% 页面设置
\pagestyle{plain}
\usepackage[a4paper, top=1in, bottom=1in, left=1.5in, right=1.5in]{geometry}
\usepackage[fontsize=12]{fontsize}
% 文档内容
\begin{document}

\title{频率特性}
\author{ecstayalive}
\maketitle

\section*{频率特性分析}
本文需要你需要提前了解如下的知识。频率特性是指线性定常系统在正弦信号的作用下,
系统的输出如输入是同频率的正弦信号。仅仅是幅值与相位不同。其中我们定义:
\begin{gather}
    A(w) = |G(jw)| \\
    \phi(w) = \angle G(jw)
\end{gather}

同时,在题目之中我们也需要画出来$G(jw)$的极坐标图与Bode图,以分析系统性能。
为了读懂之后的内容,这些基本内容必须有所了解。

\subsection*{极坐标图}
第一个问题就是我们常常需要绘制系统的开环传函极坐标图,为实现这一点,我们需要
将$G(s)$分解为一些基本环节,接下来在对这些环节进行详细的分析。

\subsection*{极坐标图}
在问题中我们常常需要绘制系统的极坐标图。为做到这一点,我们需要将系统分解为基本环节。



\end{document}
