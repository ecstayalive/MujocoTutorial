\documentclass{ctexart}
% 页面设置
\pagestyle{plain}
\usepackage[a4paper, top=1in, bottom=1in, left=1.5in, right=1.5in]{geometry}
\usepackage[fontsize=12]{fontsize}
\usepackage{amsmath, amssymb, mathtools}
% 文档内容
\begin{document}
\title{流程}
\author{ecstayalive}
\maketitle

\section*{一些流程}
虽然每个人都希望看到有很多技巧,能够帮助你加快做题速度。甚至有些人希望能够看到题目,立刻
得出来答案,但是很不幸,没有这种方法,大部分题目仍需要我们做大量的运算。
但是确实一些如果我们在作题目的时候遵循一些规范会有所帮助。本篇文章只会介绍几个应该具有的
规范问题,当然,每个人都有自己的规范、方法。在此只做提示。

\subsection*{多项式乘积}
在自动控制原理做题时,经常会遇见多项式乘积。但是题目中的多项式乘积是非常长的,
这就会导致在写的过程中十分混乱。例如:

\begin{equation}
    \begin{aligned}
        \quad   & (2 s^2 + s + 5)(a s + b)(3s + 2)                    \\
        = \quad & (2a s^3 + 2b s^2 + a s^2 + b s + 5a s + 5b)(3s + 2) \\
        = \quad & (2a s^3 + (2b + a) s^2 + (5a + b)s + 5b)(3s + 2)    \\
        = \quad & \cdots
    \end{aligned}
    \label{eq:1}
\end{equation}

由这个例子看到,当处理比较长的数学公式的时候,就显得十分混乱。这种混乱会增大我们
出错的概率。有一种方法似乎可以避免这种混乱。就是每次在处理的时候写成堆叠状,例如:
当我们处理$(2 s^2 + s + 5)(a s + b)(3s + 2)$时,
我们需要先处理$(2 s^2 + s + 5)(a s + b)$,这个时候我们写成堆叠形式:

\begin{equation}
    \begin{aligned}
        \quad   & (2 s^2 + s + 5)(a s + b)               \\
        = \quad & 2a s^3 + a s^2 + 5a s                  \\
        \quad   & \qquad \quad 2b s^2  + b s  + 5b       \\
        = \quad & 2a s^3 + (a + 2b) s^2 + (5a + b)s + 5b
    \end{aligned}
    \label{eq:2}
\end{equation}

这种对齐形式的写法会降低错误率,同时也可以稍微提升做题速度。但是尽管如此,大部分题目仍然需要
较强的计算能力。

\subsection*{现代控制理论中的一些流程}

现代控制理论中比传统控制理论涉及到更多的矩阵运算,如果在运算过程中没有一定的流程的话,
在计算过程中可能会让人迷糊。在现控中,涉及最多的是三阶矩阵的乘法,这是我们可以使用如下
的流程来做:

\begin{equation}
    \begin{pmatrix}
        a_{11} & a_{12} & a_{13} \\
        a_{21} & a_{22} & a_{23} \\
        a_{31} & a_{32} & a_{33}
    \end{pmatrix}
    \begin{pmatrix}
        b_{11} \\
        b_{21} \\
        b_{31}
    \end{pmatrix}
    =
    b_{11}
    \begin{pmatrix}
        a_{11} \\
        a_{21} \\
        a_{31}
    \end{pmatrix}
    +
    b_{21}
    \begin{pmatrix}
        a_{12} \\
        a_{22} \\
        a_{32}
    \end{pmatrix}
    +
    b_{31}
    \begin{pmatrix}
        a_{13} \\
        a_{23} \\
        a_{33}
    \end{pmatrix}
    \label{eq:3}
\end{equation}
这样,矩阵乘法可以使用三个向量加法来表示。虽然这并没有减少运算量,但至少看起来比较有条理。
如果学到现代控制系统可控、可观性判别时。会发现经常让求一个矩阵$\begin{bmatrix}
        b & Ab & A^2b & \cdots
    \end{bmatrix}$,
通常$A$矩阵为三阶。面对此矩阵时即可使用公式\eqref{eq:3},
将$Ab$的向量值算出。当$Ab$的值算出后,我们无需求解$A^2$矩阵,只需要迭代的使用公式\eqref{eq:3}即可。

就比如,我们已经通过下式计算得到了$Ab$向量:

\begin{equation}
    \begin{aligned}
        Ab& = \quad
        \begin{pmatrix}
            a_{11} & a_{12} & a_{13} \\
            a_{21} & a_{22} & a_{23} \\
            a_{31} & a_{32} & a_{33}
        \end{pmatrix}
        \begin{pmatrix}
            b_{1} \\
            b_{2} \\
            b_{3}
        \end{pmatrix}     \\
                 & = \quad
        b_{1}
        \begin{pmatrix}
            a_{11} \\
            a_{21} \\
            a_{31}
        \end{pmatrix}
        +
        b_{2}
        \begin{pmatrix}
            a_{12} \\
            a_{22} \\
            a_{32}
        \end{pmatrix}
        +
        b_{3}
        \begin{pmatrix}
            a_{13} \\
            a_{23} \\
            a_{33}
        \end{pmatrix}     \\
                 & = \quad
        \begin{pmatrix}
            c_1 \\
            c_2 \\
            c_3
        \end{pmatrix}
        \label{eq:4}
    \end{aligned}
\end{equation}

那么在我们计算下一行时,我们不需要计算$A^2$,只需要递归地使用这种向量表示法就可以。

\begin{equation}
    \begin{aligned}
        A^2b = A \cdot Ab & = \quad
        \begin{pmatrix}
            a_{11} & a_{12} & a_{13} \\
            a_{21} & a_{22} & a_{23} \\
            a_{31} & a_{32} & a_{33}
        \end{pmatrix}
        \begin{pmatrix}
            c_{1} \\
            c_{2} \\
            c_{3}
        \end{pmatrix}                                \\
                                            & = \quad
        c_{1}
        \begin{pmatrix}
            a_{11} \\
            a_{21} \\
            a_{31}
        \end{pmatrix}
        +
        c_{2}
        \begin{pmatrix}
            a_{12} \\
            a_{22} \\
            a_{32}
        \end{pmatrix}
        +
        c_{3}
        \begin{pmatrix}
            a_{13} \\
            a_{23} \\
            a_{33}
        \end{pmatrix}                                \\
                                            & = \quad
        \begin{pmatrix}
            d_1 \\
            d_2 \\
            d_3
        \end{pmatrix}
        \label{eq:4}
    \end{aligned}
\end{equation}

如果做的题目比较多,就会发现这种方法比较有条理,可以稍微地减少计算的时间,并且可以减少错误率。

\end{document}
