\documentclass{ctexart}
% 数学公式
\usepackage{amsmath, amssymb, mathtools}
%处理图片
\usepackage{graphicx}
\usepackage{float}
\usepackage{subfigure}
% 页面设置
\pagestyle{plain}
\usepackage[a4paper, top=1in, bottom=1in, left=1.5in, right=1.5in]{geometry}
\usepackage[fontsize=12]{fontsize}
% 文档内容
\begin{document}

\title{求分离点为什么要求导呢?}
\author{ecstayalive}
\maketitle

\section*{原理}

假设我们已经知道了系统的特征方程为$D(s) = 0$,那么求解该方程可以得到所有的闭环极点。
当然,此方程内部还具有一个参数$K$,我们在这里并没有写出来。但是可以确定的是,在分离
点处,至少会有两个根轨迹相交。也因此,此时$D(s) = 0$必然至少存有两个相等的根。
这也就是根轨迹在分离点处必须满足的条件。可是这与求导又有什么关系呢?

首先,你需要仔细地理解前面$D(s)$在分离点处满足的条件,那么现在让我们揭示这两者之间的关系。
首先$D(s)$是内部函数多个参数,关于$s$的多项式方程,只不过通常,我们遇见的$D(s)$只含有
一个参数$K$。既然$D(s)$是一个多项式方程,那么$D(s)$就可以写成如下的形式:

\begin{equation}
    D(s) = (s-p_1)(s-p_2)\cdots(s-p_{n-1})(s-p_n) = 0
\end{equation}

其中$p_1, p_2, \cdots, p_n$是系统的极点。由于在分离点处$D(s)$比然有两个重根,也因此
这n个极点中必然至少有两个相等的。不妨就设在这n个极点中$p_i=p_j$。但是,本文为了方便理解,
便不引入过多的变量。就设$p_1=p_2$。那么明显$D(p_1) = D(p_2) = 0$,在点$p_1$处符合
分离点的条件。此时,我们计算一下$D(s)$的导函数:

\begin{equation}
    \begin{aligned}
        \frac{d D(s)}{s} = & (s-p_2)\cdots(s-p_n) + (s-p_1)\cdots(s-p_n) +\cdots \\
                           & + (s-p_1)(s-p_2)\cdots(s-p_{n-1})
    \end{aligned}
\end{equation}

明显在重极点$p_1$处其导函数等于0。

\section*{使用范围}
之前已经讲了为什么在分离点处要求导。但是并不是所有的题目都用求导,因为由于参数$K$的存在,
有时候求导没有效果。同时,课本上已经给出了一个求分离点公式。但是$G(s)$符合如下的情况,
我们可以直接对特征方程进行求导:
\begin{equation}
    G(s) = \frac{K^*}{(s-p_1)\cdots(s-p_n)}
\end{equation}

此时我们可以对$D(s)$直接求导,求零点。因为此时有:
\begin{gather}
    D(s) = s^n + a_{n-1}s^{n-1} + \cdots + a_0 + K^* \\
    D^{'}(s) =n s^{n-1} + \cdots + a_1
\end{gather}

即$D^{'}(s)$不会受未知参数的影响。

\end{document}
