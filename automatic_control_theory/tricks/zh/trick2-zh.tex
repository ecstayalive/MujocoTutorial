\documentclass{ctexart}

% 页面设置
\pagestyle{plain}
\usepackage[a4paper, top=1in, bottom=1in, left=1.5in, right=1.5in]{geometry}
\usepackage[fontsize=12]{fontsize}

\begin{document}

\title{线性离散时间控制系统分析与综合}
\author{ecstayalive}
\date{2023年6月15日}
\maketitle

\section{一个很重要的公式}
相对而言,离散控制系统往往计算量比较大,尤其是在不允许使用计算机的情况下。
而如果想要做题速度快,就需要多记。因此本文介绍一个十分重要的公式,使用该公式
能够较快的提高做某些离散控制题的速度。

\begin{equation}
    \mathbf{Z}[\frac{a^2}{s^2(s+a)}] = \frac{z[(aT+e^{-aT}-1)z +
    1-(aT+1)e^{-aT}]}{(z-1)^2(z-e^{-aT})}
\end{equation}

可以发现,这个公式比较复杂,因此在教材中一般不会说明。但是如果有做到离散系统
这部分的同学会发现让求$\frac{a^2}{s^2(s+a)}$这类形式的$\mathbf{Z}$变换
频率极高。

仔细观察一下这个式子,会发现其中只有$e^{-aT}$与$aT$这两个需要较大的计算量,
一般而言,题目会告诉你采样周期$T$, 如果记住$e^{-1} = 0.368$等几个常用的式子
的话,整体就十分好算。因此也能极大地提高做题速度。

\subsection{离散系统中推荐记住的几个复杂的公式}
接下来这几个公式往往教材中都有涉及

\begin{equation}
    \mathbf{Z}[\frac{a}{s(s+a)}] =
    \frac{z(1 - e^{-aT})}{(z-1)(z-e^{-aT})}
\end{equation}

\begin{equation}
    \mathbf{Z}[\frac{b-a}{(s+a)(s+b)}] =
    \frac{z(e^{-aT}-e^{-bT})}{(z-e^{-aT})(z-e^{-bT})}
\end{equation}

除此之外,课本上比较简单、基础的$\mathbf{Z}$变换公式必须记住。

\end{document}
